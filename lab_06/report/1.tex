\chapter{Лабораторная работа №6}

\textbf{Смирнов Пётр, ИУ7-65Б}

\section{Номер 1}

Написать хвостовую рекурсивную функцию my-reverse, которая развернёт верхний
уровень своего списка-аргумента lst.

\begin{figure}[H]
    \begin{listingbox}{}
        \lstinputlisting[language=Lisp]{../1.lisp}
    \end{listingbox}
    \label{lst:1}
\end{figure}

\section{Номер 2}

Написать функцию, которая возвращает первый элемент списка-аргумента, который 
сам является непустым списком.

\begin{figure}[H]
    \begin{listingbox}{}
        \lstinputlisting[language=Lisp]{../2.lisp}
    \end{listingbox}
    \label{lst:2}
\end{figure}

\section{Номер 3}

Напишите рекурсивную функцию, которая умножает на заданное число-аргумент 
все числа из заданного списка-аргумента, когда 

а) все элементы списка -- числа;

\begin{figure}[H]
    \begin{listingbox}{}
        \lstinputlisting[language=Lisp]{../3-a.lisp}
    \end{listingbox}
    \label{lst:3}
\end{figure}

б) элементы списка -- любые объекты.

\begin{figure}[H]
    \begin{listingbox}{}
        \lstinputlisting[language=Lisp]{../3-b.lisp}
    \end{listingbox}
    \label{lst:3}
\end{figure}

\section{Номер 4}

Напишите функцию select-between, которая из списка-аргумента, содержащего
только числа, выбирает только те, которые расположены между двумя указанными 
границами-аргументами и возвращает их в виде списка, упорядоченного по возрастанию.
(+ 2 балла)

\begin{figure}[H]
    \begin{listingbox}{}
        \lstinputlisting[language=Lisp]{../4.lisp}
    \end{listingbox}
    \label{lst:4}
\end{figure}

\section{Номер 5}

Написать рекурсивную версию (с именем rec-add) вычисления суммы чисел заданного
списка:

а) одноуровнего смешанного;

\begin{figure}[H]
    \begin{listingbox}{}
        \lstinputlisting[language=Lisp]{../5-a.lisp}
    \end{listingbox}
\end{figure}

б) структурированного.

\begin{figure}[H]
    \begin{listingbox}{}
        \lstinputlisting[language=Lisp]{../5-b.lisp}
    \end{listingbox}
\end{figure}

\section{Номер 6}

Написать рекурсивную версию с именем recnth функции nth.

\begin{figure}[H]
    \begin{listingbox}{}
        \lstinputlisting[language=Lisp]{../6.lisp}
    \end{listingbox}
    \label{lst:6}
\end{figure}

\section{Номер 7}

Написать рекурсивную функцию allodd, которая возвращает t, когда все элементы
списка нечётные.

\begin{figure}[H]
    \begin{listingbox}{}
        \lstinputlisting[language=Lisp]{../7.lisp}
    \end{listingbox}
    \label{lst:7}
\end{figure}

\section{Номер 8}

Написать рекурсивную функцию, которая возвращает первое нечётное число из списка
(структурированного), возможно создавая некоторые вспомогательные функции.

\begin{figure}[H]
    \begin{listingbox}{}
        \lstinputlisting[language=Lisp]{../8.lisp}
    \end{listingbox}
    \label{lst:8}
\end{figure}

\section{Номер 9}

Используя cons-дополняемую рекурсию с одним тестом завершения, 
написать функцию, которая получает как аргумент список чисел, а возвращает
список квадратов этих чисел в том же порядке.

\begin{figure}[H]
    \begin{listingbox}{}
        \lstinputlisting[language=Lisp]{../9.lisp}
    \end{listingbox}
    \label{lst:9}
\end{figure}

\section{Номер 10}

Преобразовать структурированный список в одноуровневый.

\begin{figure}[H]
    \begin{listingbox}{}
        \lstinputlisting[language=Lisp]{../10.lisp}
    \end{listingbox}
    \label{lst:10}
\end{figure}