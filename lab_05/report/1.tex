\chapter{Лабораторная работа №5}

\textbf{Смирнов Пётр, ИУ7-65Б}

\section{Номер 1}

Напишите функцию, которая уменьшает на 10 все числа из списка-аргумента этой
функции, проходя по верхнему уровню списковых ячеек. (список смешанный 
структурированный)

\begin{figure}[H]
    \begin{listingbox}{}
        \lstinputlisting[language=Lisp]{../1.lisp}
    \end{listingbox}
    \label{lst:1}
\end{figure}

\section{Номер 2}

Написать функцию, которая получает как аргумент список чисел, а возвращает
список квадратов этих чисел в том же порядке.

\begin{figure}[H]
    \begin{listingbox}{}
        \lstinputlisting[language=Lisp]{../2.lisp}
    \end{listingbox}
    \label{lst:2}
\end{figure}

\section{Номер 3}

Напишите функцию, которая умножает на заданное число-аргумент все числа из
заданного списка-аргумента, когда 

а) все элементы списка -- числа;

б) элементы списка -- любые объекты.

\begin{figure}[H]
    \begin{listingbox}{}
        \lstinputlisting[language=Lisp]{../3-a.lisp}
    \end{listingbox}
    \label{lst:3}
\end{figure}

\begin{figure}[H]
    \begin{listingbox}{}
        \lstinputlisting[language=Lisp]{../3-b.lisp}
    \end{listingbox}
    \label{lst:3}
\end{figure}

\section{Номер 4}

Написать функцию, которая по своему списку-аргументу lst определяет, является
ли он палиндромом для одноуровнего смешанного списка.

\begin{figure}[H]
    \begin{listingbox}{}
        \lstinputlisting[language=Lisp]{../4.lisp}
    \end{listingbox}
    \label{lst:4}
\end{figure}

\section{Номер 5}

Написать предикат set-equal, который возвращает t, если два его множества-аргумента
(одноуровневые списки) содержат одни и те же элементы, порядок которых не имеет 
значения.

\begin{figure}[H]
    \begin{listingbox}{}
        \lstinputlisting[language=Lisp]{../5.lisp}
    \end{listingbox}
    \label{lst:5}
\end{figure}

\section{Номер 6}

Напишите функцию select-between, которая из списка-аргумента, содержащего
только числа, выбирает только те, которые расположены между двумя указанными числами
-- границами-аргументами и возвращает их в виде списка, упорядоченного по возрастанию.
(+ 2 балла)

\begin{figure}[H]
    \begin{listingbox}{}
        \lstinputlisting[language=Lisp]{../6.lisp}
    \end{listingbox}
    \label{lst:6}
\end{figure}

\section{Номер 7}

Написать функцию, вычисляющую декартово произведение двух своих списков-аргументов.

\begin{figure}[H]
    \begin{listingbox}{}
        \lstinputlisting[language=Lisp]{../7.lisp}
    \end{listingbox}
    \label{lst:7}
\end{figure}

\section{Номер 8}

Почему так реализовано reduce, в чём причина?

\begin{figure}[H]
    \begin{listingbox}{}
        \lstinputlisting[language=Lisp]{../8.lisp}
    \end{listingbox}
    \label{lst:8}
\end{figure}

Если список, подаваемый на вход пуст, то функция, указанная вторым 
аргументом reduce вызывается без аргументов. +, вызываемый без аргументов
возвращает 0. *, вызываемая без аргументов, возвращает 1.

\section{Номер 9}

Пусть list-of-list список, состоящий из списков. Написать функцию, которая 
вычисляет сумму длин всех элементов list-of-list (количество атомов).

\begin{figure}[H]
    \begin{listingbox}{}
        \lstinputlisting[language=Lisp]{../9.lisp}
    \end{listingbox}
    \label{lst:9}
\end{figure}