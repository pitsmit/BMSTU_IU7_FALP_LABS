\chapter{Лабораторная работа №3}

\textbf{Смирнов Пётр, ИУ7-65Б}

\section{Номер 1}

Написать функцию, которая принимает целое число и возвращает 
первое чётное число, не меньшее аргумента.

\begin{figure}[H]
    \begin{listingbox}{}
        \lstinputlisting[language=Lisp]{../1.lisp}
    \end{listingbox}
    \label{lst:1}
\end{figure}

\section{Номер 2}

Написать функцию, которая принимает число и возвращает 
число того же знака, но с модулем на 1 больше модуля аргумента.

\begin{figure}[H]
    \begin{listingbox}{}
        \lstinputlisting[language=Lisp]{../2.lisp}
    \end{listingbox}
    \label{lst:2}
\end{figure}

\section{Номер 3}

Написать функцию, которая принимает два числа и возвращает
список из этих чисел, расположенных по возрастанию.

\begin{figure}[H]
    \begin{listingbox}{}
        \lstinputlisting[language=Lisp]{../3.lisp}
    \end{listingbox}
    \label{lst:3}
\end{figure}

\section{Номер 4}

Написать функцию, которая принимает три числа и возвращает T
тогда, когда первое число расположено между вторым и третьим.

\begin{figure}[H]
    \begin{listingbox}{}
        \lstinputlisting[language=Lisp]{../4.lisp}
    \end{listingbox}
    \label{lst:4}
\end{figure}

\section{Номер 5}

Каков результат вычисления следующих выражений?

\begin{figure}[H]
    \begin{listingbox}{}
        \lstinputlisting[language=Lisp]{../5.lisp}
    \end{listingbox}
    \label{lst:5}
\end{figure}

\section{Номер 6}

Написать предикат, который принимает два числа-аргумента 
и возвращает T, если первое число не меньше второго.

\begin{figure}[H]
    \begin{listingbox}{}
        \lstinputlisting[language=Lisp]{../6.lisp}
    \end{listingbox}
    \label{lst:6}
\end{figure}

\section{Номер 7}

Какой из следующих двух вариантов предиката ошибочен и почему?

\begin{figure}[H]
    \begin{listingbox}{}
        \lstinputlisting[language=Lisp]{../7.lisp}
    \end{listingbox}
    \label{lst:7}
\end{figure}

Второй ошибочен, так как там проверка что это числовой атом идет после
проверки на положительность.

\section{Номер 8}

Решить задачу 4, используя для её решения конструкции:
только IF, только COND, только AND/OR.

\begin{figure}[H]
    \begin{listingbox}{}
        \lstinputlisting[language=Lisp]{../8.lisp}
    \end{listingbox}
    \label{lst:8}
\end{figure}

\section{Номер 9}

Переписать функцию how-alike, приведённую в лекции и использующую
COND, используя только конструкции IF, AND/OR

\begin{figure}[H]
    \begin{listingbox}{}
        \lstinputlisting[language=Lisp]{../9.lisp}
    \end{listingbox}
    \label{lst:9}
\end{figure}

\clearpage

\begin{figure}[H]
    \begin{listingbox}{}
        \lstinputlisting[language=Lisp]{../defence.lisp}
    \end{listingbox}
    \label{lst:defence}
\end{figure}